\documentclass{article}
\usepackage{graphicx} % Required for inserting images

% to make itemizes take less space
\usepackage{enumitem}% http://ctan.org/pkg/enumitem

\title{GPIG-A Formative}
\begin{document}
\maketitle

\section{An outline of the proposed technical solution}

We are concerned with the growth of destructive plants within a forest, these plants can outcompete their neighbours, and reduce biodiversity. 
We propose a solution that uses drones to monitor and capture image data of a given forest, each image is handed back to a central system where we identify the plants present in the image, and store that along with the latitude and longitude. 
This system is generic enough to work in any forest, but has the option to train it against specific plants that you are concerned about, such that it can give deeper insights and recommendations.
We will have a user interface where an operator can:
\begin{itemize}[noitemsep,topsep=3pt]
\item See the locations of each drone,
\item Command the cluster of drones to move to a new location,
\item See which areas have most recently been scanned,
\item Chat with a large language model about the data obtained.
\end{itemize}


The plants within each captured image will be clustered together such that similar looking plants are given the same ID. 
For each plant species that we are especially concerned with, $p$, we could train a plant classification module $f(s) -> [0,1]$ mapping a series of images, $s$ belonging to a given id onto the probability that they are of species $p$. 
This would give us reasonable confidence that our id maps to a given species.

By tracking the quantity of each identified plant over time, we can see if any are undergoing exponential growth.
Using a button on our user interface, this data will get wrapped into a pre-made prompt that can then be sent to a large language model.
The pre-made prompt would say which plants are growing and where, either the plant id or the predicted plant name, the drone scan history, the location of the forest, and ask for advice on how to handle it.
The operator can then have a conversation with the large language model where they try to plot the best course of action going forward.


\section{A brief description of the system that will be prototyped (including assumptions, capabilities, primary functions)}

% TODO, not really sure how to talk about assumptions, capabilities, and primary functions. I have done my best but this could definitely do with some major work.

The following systems will be prototyped:
\begin{itemize}[noitemsep,topsep=3pt]
\item The datastore and data analysis,
\item The plant clustering and id assignment,
\item An example plant classifier module,
\item The user interface,
\item Drone simulation,
\item Dummy data feeds,
\item Large language model communication,
\item The system will be containerised.
\end{itemize}

\noindent
The datastore will be a PostgreSQL database containing three tables:
\begin{itemize}[noitemsep,topsep=3pt]
\item Raw image entries containing columns for latitude, longitude, camera pose, and the corresponding image URI,
\item Processed entries containing columns for latitude, longitude, and plant id,
\item A map from plant id to species.
\end{itemize}
We assume that this should be sufficient for the final product.

% This paper is due in april, I suspect that we will know by then so maybe this will need changing in the future. 
% TODO Babar should expand on this
We aren't currently sure how exactly we will do the plant clustering and id assignment.
If using k-means clustering we would have to carefully define a value of K for each forest, this wouldn't be ideal as it reduces the genericness of our system.
We will need to look into clustering techniques that work with variable cluster quantities.
Our plant classifier module will take a cluster of images and predict whether they belong to the species Japanese knotweed, to do this we will need to gather a large number of images of the plant to use as training data. We assume that this will be possible. % by the submission of this paper we will likely want to change the tense of this

% TODO Maybe marks and ellie can expand on this
The user interface will be created using react and we will prioritise the functionality over UX/UI design. 

% TODO Maybe Jacob and Jakub can work on this
Drone simulation will be done using ROS2 and Unreal Engine %TODO Is this correct?
We assume that the code in the simulation should run on a real robot in a forest, but this will need to be tested and verified as simulation is only an approximation of the real world and there are many factors that we can't account for.
Because of this we will use dummy data feeds to simulate what the drones would send to the actual system.
Without significant effort, the images captured in simulation won't be similar to the ones we would capture in real life, therefore, we will create a fake node that can deliver data to our datastore.

% TODO Maybe Babar and CJ can expand on this
Our large language model communication will be handled as follows: Take the data from our system, format it into a prompt, send that to an LLM, forward response to user interface.
In the final product this could use a specially trained LLM or a fine-tuned LLM, but for our prototype we will hook into GPT-3.5 using their public API % we might not actually use this LLM, change if necessary. 

% TODO Maybe Laura can expand on this
We have containerised our prototype so that functionality is split into microservices, these can be worked on independently by each team.
Each microservice is containerised and the entire project can be ran using docker compose.

% TODO maybe we should talk about our sizing estimations, once Prajj has completed his sizing spreadsheet he could talk about it here?

\section{An overview of the system aspects that the team intends to demonstrate in the group}
presentation
    Show that we can move drones
    Show what happens when plants grow destructively
        Simulation part
        Data pipeline 
    Chatting with the LLM


\section{A summary of progress}


\section{A brief risk register}

\end{document}
